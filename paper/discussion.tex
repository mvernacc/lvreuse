\documentclass[conf]{new-aiaa}
%\documentclass[journal]{new-aiaa} for journal papers
\usepackage[utf8]{inputenc}

\usepackage{graphicx}
\usepackage{amsmath}
\usepackage[version=4]{mhchem}
\usepackage{siunitx}
\usepackage{longtable,tabularx}
\usepackage{footnote}
\usepackage{mhchem}
\usepackage{physics}
\usepackage{array,makecell,booktabs}
\newcolumntype{M}[1]{>{\centering\arraybackslash}m{#1}}
\usepackage[super]{nth}
\makesavenoteenv{tabular}
\setlength\LTleft{0pt}

\graphicspath{{figures/}}

\begin{document}

\section{Discussion}

This section presents estimates of the cost-per-flight of various reuse strategies. Two strategies, (downrange, rocket-propelled, propulsive landing, full recovery) and (downrange, no propulsion, parachute, full recovery), are shown to dominate the other strategies on cost. However, parachute recovery of the full first stage is likely only practical for small launch vehicles\footnote{A small stage on parachutes could be recovered in midair or on a ship, or survive landing on land. A large stage on parachutes may need to land directly in the ocean, which would increase recovery and refurbishment expenses.}, and it will also be shown that first stage reuse is not favorable for small launch vehicles. Thus, downrange propulsive landing appears to be the dominant strategy.

Launch rate is also a critical factor in the economics of reuse. Increasing launch rate further reduces cost-per-flight, and also allows the investment in reuse development to be paid off more quickly. Efficient first stage reuse may allow for an increase in launch rate. A market that can support a high (>20 flights/year) launch rate may be a critical factor for the economic viability of a first stage reuse.

The discussion is organized as follows: the first subsection shows the distribution of cost-per-flight estimates under a generic dispersion of the model input parameters. The subsequent subsections examine the effect of number of reuses, launch rate, and launch vehicle size on cost. Finally, the last section discusses whether the present value of cost savings from reuse is enough to justify the initial investment in its development.

\subsection{Strategy Cost-per-flight Estimates}
violin plots of strategy cpfs
cpf varies with technology choice, mission, large uncertainties with winged vehicles

\subsection{Effect of Number of Reuses and Launch Rate}
stack plot with reuse sweep- show why cost declines with number of reuses
cpf vs. reuse plot with varying launch rate curves - shows significant decrease in cost with increasing launch rate for low launch rates
cpf vs. reuse plot for different strategies 

\subsection{Effect of Launch Vehicle Size}
reuse doesn't make sense for smaller vehicles - costs not dominated by hardware cost

\subsection{Paying off development costs}
npv plot

\bibliography{first_stage_recovery}

\end{document}