\documentclass[conf]{new-aiaa}
%\documentclass[journal]{new-aiaa} for journal papers
\usepackage[utf8]{inputenc}

\usepackage{graphicx}
\usepackage{amsmath}
\usepackage[version=4]{mhchem}
\usepackage{siunitx}
\usepackage{longtable,tabularx}
\usepackage{footnote}
\usepackage{mhchem}
\usepackage{physics}
\usepackage{array,makecell,booktabs}
\newcolumntype{M}[1]{>{\centering\arraybackslash}m{#1}}
\usepackage[super]{nth}
\makesavenoteenv{tabular}
\setlength\LTleft{0pt}

\graphicspath{{figures/}}

\begin{document}

\section{Performance Model}

Overall payload mass fraction is the key performance metric

$\pi_*$ predicted by the rocket equation
    reveiw/restate rocket equation
    show model calibration against Delta IV and Falcon 9 $pi_*$ vs. $\Delta v_*$

\subsection{Technology choice}
define
clearly performance depends on min inert mass and Isp, but that's not what we're her to talk about...

\subsection{How first stage reuse effects performance}
How does reuse change the rocket equation?
    dv losses - minor
    mass on first stage - major

\subsection{First stage unavailable mass}
define

\subsection{Relating unavailable mass to recovery hardware and propulsion}
define

\subsection{Estimating hardware and propulsion for recovery strategies}
introduce/review Monte Carlo method
refer to appendix for derivation of input distributions
show e vs P, a contour plot

\subsection{Payload fraction estimates for recovery strategies}
show violin plots

\bibliography{first_stage_recovery}

\end{document}