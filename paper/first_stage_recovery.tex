\documentclass[conf]{new-aiaa}
%\documentclass[journal]{new-aiaa} for journal papers
\usepackage[utf8]{inputenc}

\usepackage{graphicx}
\usepackage{amsmath}
\usepackage[version=4]{mhchem}
\usepackage{siunitx}
\usepackage{longtable,tabularx}
\usepackage{footnote}
\makesavenoteenv{tabular}
\setlength\LTleft{0pt} 

\title{Preparation of Papers for AIAA Technical Conferences}

\author{First A. Author\footnote{Insert Job Title, Department Name, Address/Mail Stop, and AIAA Member Grade (if any) for first author.} and Second B. Author Jr.\footnote{Insert Job Title, Department Name, Address/Mail Stop, and AIAA Member Grade (if any) for second author.}}
\affil{Business or Academic Affiliation 1, City, State, Zip Code}
\author{Third C. Author\footnote{Insert Job Title, Department Name, Address/Mail Stop, and AIAA Member Grade (if any) for third author.}}
\affil{Business or Academic Affiliation 2, City, Province, Zip Code, Country}
\author{Fourth D. Author\footnote{Insert Job Title, Department Name, Address/Mail Stop, and AIAA Member Grade (if any) for fourth author (etc.).}}
\affil{Business or Academic Affiliation 2, City, State, Zip Code}

\begin{document}

\maketitle

\begin{abstract}
These instructions give you guidelines for preparing papers for AIAA Technical Papers using \LaTeX{}. Define all symbols used in the abstract. Do not cite references in the abstract. The footnote on the first page should list the Job Title and AIAA Member Grade for each author, if known Authors do not have to be AIAA members.
\end{abstract}

\section{Description of Recovery Strategies}
TODO
\begin{table}
	\caption{\label{tab:vehicles} Some launch vehicles with reusable boosters. (* denotes boosters used in parallel staging)}
	\centering
	\begin{tabular}{p{4cm} l p{2cm}  p{2cm} p{2cm}}
		\hline
		Vehicle & Status & Portion of booster recovered & Deccel \& Manuever method & Recovery location \\
		\hline
		\hline
		Falcon 9 & Operational & Full & Propulsive & Launch site or downrange \\
		\hline
		Space Shuttle* & Retired & Full & Parachute & Downrange \\
		\hline
		New Glenn & Proposed & Full & Propulsive & Downrange \\
		Adeline (Ariane 6) & Proposed & Partial & Wing & Launch site \\
		SMART (Vulcan) \cite{Ragab2015} & Proposed & Partial & Parachute & Downrange midair \\
		\hline
		Ares I \cite{Ares2009} & Canceled & Full & Parachute & Downrange \\
		Reusable Booster System (RBS) \cite{NAP13534} & Canceled & Full & Propulsive \& Wing & Launch site\\
		NASA Liquid Fly-Back Booster (LFBB)* \cite{Healy1998} & Canceled & Full & Propulsive \& Wing & Launch site\\
		DRL Liquid Fly-Back Booster (LFBB)* \cite{Sippel2003} & Canceled & Full & Propulsive \& Wing & Launch site\\
		\hline
	\end{tabular}
\end{table}


\section{Performance Model}
TODO

\section{Cost Model}
Next, we attempt to quantify the cost savings from reusing the first stage. The cost savings are quantified by the cost factor $r_c$, which is the ratio of the average cost of one flight of a reusable first stage to the average cost of an expendable first stage. For $r_c < 1$, the reusable system has lower recurring costs. Write $r_c$ in terms of some of the cost elements identified in \cite{Sforza2015}:

\begin{equation}
\label{eq:cost_elements}
r_c = \frac{\frac{C_{pr}}{n} + C_{pn} + C_f + C_{ro}}{C_{pe}}
\end{equation}

where $n$ is the average lifetime number of flights for a reusable first stage. The cost elements of the reusable first stage are: $C_{pr}$, the production cost of the reusable components; $C_{pn}$, the production cost of the new (non-reusable) components; $C_f$, the refurbishment cost; and $C_{ro}$, the recovery operations cost. The expendable stage production cost is $C_{pe}$. Note that this cost model does not capture development costs. Also, $r_c$ only includes those costs directly relevant to the first stage, and excludes other factors which contribute to the launch service cost (upper stage, facilities, and insurance, etc.).

Actual costs vary with booster size scale, are difficult to estimate for new vehicle proposals \cite{Sforza2015}, and are often not publicly available. Instead, it is more helpful to write the cost factor in terms of dimensionless ratios which can be more readily estimated. I define the following ratios:
\begin{itemize}
	\item The recovered cost ratio, $z$, which is the ratio of the cost of the recovered hardware to the total production cost of the recoverable stage
    \begin{equation}
    z = \frac{C_{pr}}{C_{pr} + C_{pn}}
    \end{equation}
    By definition, $z \in [0, 1]$. For full-recovery strategies, $z=1$. For strategies which recover only the engines and other high-value components, $z \approx 0.5$ \cite{Ragab2015} (TODO more general engine cost as fraction of stage cost. I think Atlas V has a unusually expensive engine).
    
    \item The recovery and refurbishment cost ratio, $q$, which is the cost of recovery and refurbishment divided by production cost of the reusable hardware.
    \begin{equation}
    q = \frac{C_{f} + C_{ro}}{C_{pr}}
    \end{equation}
    By definition, $q > 0$. Inefficient recovery strategies can have $q > 1$ (e.g. SRB citation needed), in which case recovery cannot lead to cost savings and is probably not worthwhile. In some extreme proposals, the only recovery/refurbishment operation would be to reload the stage with propellant \cite{Musk2017}. In this case, propellant costs give a lower limit of $q \approx 0.01$ for hydrocarbon/oxygen stages \cite{Ragab2015}. A more moderate estimate from \cite{Sforza2015} gives $q \approx 0.25$ (15\% for refurbishment and another 10\% for recovery operations). Refurbishment cost data for Falcon 9 is not publicly available, but SpaceX leadership has stated that the refurbishment cost of the first re-used stage was ``substantially less than half'' of the production cost ($q<0.5$) \cite{Foust2017}.
   
    \item The production cost ratio, $b$, which is the ratio of production costs for reusable and expendable stages.
    \begin{equation}
    b = \frac{C_{pr} + C_{pn}}{C_{pe}}
    \end{equation}
    We should expect that, all other factors being equal, reusable stages will be more expensive to produce, so $b>1$. Several factors increase the cost of reusable stages:
    \begin{itemize}
    	\item Extra hardware - Recovery devices (which are not required on an expendable version) add to the production cost. The magnitude of this effect is difficult to estimate in a general manner.
    	\item More durable hardware - More expensive materials or designs may be employed to extend component life. The magnitude of this effect may increase weakly with $n$, but is  difficult to estimate in a general manner.
    	\item Lower production rate - If a reusable and expendable fleet are to provide the same total number of launches \footnote{This may not be a good assumption - if a fleet has significantly lower launch costs, it may be able to capture a larger share of the launch market or increase the volume of the launch market}, the expendable fleet will require $n$ times more first stages. Due to experience curve effects, the average cost of producing a stage is expected to decline as the production volume increases. The average reusable stage production cost would therefore be higher, even if the first-unit costs are identical. If the experience curve is described by a power law with learning rate $s$, then
    	\begin{equation}
    	b = \frac{C_{pr}' n^{-\log_2(s)} + C_{pn}'}{C_{pe}'}
    	\end{equation}
    	where primes denote first-unit costs (as opposed to average costs).
    \end{itemize}
\end{itemize}

Using these factors, Equation \ref{eq:cost_elements} can be re-written as

\begin{equation}
\label{eq:r_c_dimensionless}
r_c = z b \left( \frac{1}{n} + \frac{1 - z}{z} + q \right)
\end{equation}

There is substantial uncertainty and project-to-project variation in the cost ratios $b$ and $q$, so we should not expect to predict a single, general value for $r_c$. However, we can establish some plausible bounds on $r_c$ for the full and partial reuse cases. In the limit of a ``perfectly reusable'' stage ($n\rightarrow\infty, b=1, q=0$), Equation \ref{eq:r_c_dimensionless} reduces to $r_c = 1 - z$. This puts a hard lower limit on the cost factor for partial reuse strategies:

\begin{equation}
\label{eq:r_c_low_limit}
r_c > 1 - z
\end{equation}

The cost model is further explored in Figure \ref{fig:cost_model}, which shows contours of $r_c$ over a plausible range of $q$ and $b$. Each subplot shows an different scenario for the recovered cost ratio $z$ and the component life $n$. Several interesting factors are apparent. First, partial recovery strategies ($z=0.5$, right column of plots) are more sensitive to changes in the production cost ratio $b$, and are impractical for $b>2$ even with long life and low refurbishment costs. Second, full recovery strategies can offer large 1st stage cost savings (low $r_c$) if the refurbishment costs are kept low and the stages have long lifetimes, even if production costs for reusable stages are relatively high. In a full-recovery strategy, it is likely worthwhile to make design changes which can streamline refurbishment and increase life, even if they increase the production cost.

\begin{figure}[hbt!]
	\centering
	\includegraphics[width=1\textwidth]{../cost_model}
	\label{fig:cost_model}
	\caption{First stage re-use cost factor $r_c$ as a function of the recovery and refurbishment cost ratio $q$ and the production cost ratio $b$. Impractical regimes where re-use would not reduce cost ($r_c>1$) are shaded grey. The plots in the left column correspond to full recovery strategies ($z=1$), and the right to partial recovery ($z=0.5$). The top row corresponds to a conservative reusable component life ($n=4$ flights), the middle to a moderate life ($n=16$) and the bottom to an optimistic life ($n=64$).}
\end{figure}

To facilitate later comparisons of full vs partial recovery strategies, I compute their cost ratios in two scenarios: one making optimistic assumptions about the other variables in the reuse cost model, and one making more moderate assumptions. Table \ref{tab:cost_scenarios} shows that we can reasonably expect the cost factor to be between $0.05$ and $0.52$ for full recovery, and $0.78$ or more for partial recovery. The moderate-assumption case agrees with other assessments that a fully reusable first stage can reduce launch costs "by a factor of 2 to 3" \cite{Hampsten2010}.

\begin{table}
\caption{\label{tab:cost_scenarios} Cost factor $r_c$ in hypothetical reuse scenarios}
\centering
\begin{tabular}{lcc}
\hline
& Optimistic & Moderate \\
& $q=0.02, b=1.5, n=64$ & $q=0.20, b=2.0, n=16$ \\
\hline
Full recovery $z=1$ & 0.05 & 0.53 \\
Partial recovery $z=0.5$ & 0.78 & 1.3 \\
\hline
\end{tabular}
\end{table}


\section{Cost \& Performance Tradeoffs/Comparison}
TODO
Pareto set of best strategies.


\bibliography{first_stage_recovery}

\end{document}