\documentclass[conf]{new-aiaa}
%\documentclass[journal]{new-aiaa} for journal papers
\usepackage[utf8]{inputenc}

\usepackage{graphicx}
\usepackage{amsmath}
\usepackage[version=4]{mhchem}
\usepackage{siunitx}
\usepackage{longtable,tabularx}
\usepackage{footnote}
\usepackage{mhchem}
\usepackage{physics}
\usepackage{array,makecell,booktabs}
\newcolumntype{M}[1]{>{\centering\arraybackslash}m{#1}}
\usepackage[super]{nth}
\makesavenoteenv{tabular}
\setlength\LTleft{0pt} 

\graphicspath{{figures/}}

\title{Strategies for Re-Use of Launch Vehicle First Stages}

\author{Matthew T. Vernacchia \footnote{Research Assistant, Department of Aeronautics and Astronautics, 77 Massachusetts Avenue, AIAA Student Member.}
and Kelly J. Mathesius  \footnote{Research Assistant, Department of Aeronautics and Astronautics, 77 Massachusetts Avenue, AIAA Student Member.}}
\affil{Massachusetts Institute of Technology, Cambridge, MA, 02139}


\begin{document}

\maketitle

\section{Introduction}

Reduction of launch costs and increase of launch rates are essential for increasing accessibility to space. Many high-count commercial satellite constellations are under development \cite{SIA2017, Henry2017}, and militaries have indicated interest in distributing their space capabilities across higher-count constellations that can be quickly replenished \cite{DARPA_Blackjack}. These trends in the satellite market indicate that launch providers may need to reduce costs and increase launch rates to remain competitive. Customers may become more sensitive to launch costs as serial production drops the cost of satellites, and high-count constellations will require high launch rates to be deployed in a viable timeframe.

Dating back to von Braun, launch vehicle reusability has been proposed as a means to reduce costs and increase launch rate \cite{vonBraun52}. The essential argument has been that the high costs of launch are driven by the difficulty of producing and testing rocket hardware. Reuse would enable this cost to be spread over many flights, thereby reducing the cost per flight. Some proposals have also argued that a reusable vehicle could streamline launch operations, reducing operational costs as well \cite{Butrica03}.

However, fully reusable launch vehicles have yet to be achieved, and are difficult to develop. Several efforts have been abandoned during development (e.g. U.S. National Aerospace Plane, X-33, Delta Clipper) or de-scoped to a partially reusable system (U.S. Space Shuttle). These concepts often relied on advanced (i.e. low TRL) propulsion or structural technologies, which made development difficult and expensive.

Partial reuse of multi-stage launch vehicles is a more viable step towards cost reductions in the near term. Recovering and reusing the first stage, or part of the first stage, is considerably easier than recovering upper stages. Further, the first stage typically embodies the majority of the launch vehicle production cost. Reusing only the first stage (or part thereof) captures most of the potential economic benefit of reusability at a lower level of technical difficulty. Thus, first stage reuse is gaining considerable traction in the launch industry: SpaceX's Falcon 9 often operates with a reusable first stage, and at least three other orbital launch vehicles with first stage reuse are under serious development (Blue Origin's New Glenn, ULA's Vulcan/SMART, and Boeing's XS-1 Phantom Express).

Despite the recent surge of interest, first stage reuse is not a new idea - proposals were made as early as 1963 \cite{Nexus, SeaDragon}. Since then, a wide range of first stage reuse strategies have been proposed, differing in the manner and location of first stage recovery, and in the portion of the first stage recovered.

Observers of the field, or those embarking on a vehicle development project, may well wonder which, if any, of the first stage recovery strategies can reduce launch costs compared to contemporary expendable launch vehicles. There is still substantial disagreement on the economic merits of reuse amongst industry leaders \cite{Cantrell17, Russell18, Selding16_orbital, Wall15, Selding16_spacex}. The doubters are not without merit: adding reuse capability increases the complexity of a launch vehicle (thus increasing its development and production costs) and decreases its payload capacity. It is not immediately obvious if the savings from reusing first stage hardware outweigh these downsides. Thus, we may wonder: are any of the reuse strategies worthwhile? And if so, which is best suited to a particular segment of the launch market? Answering these questions requires a comparative analysis of the performance and cost of various first stage reuse strategies. While many first stage reuse concepts have been studied individually, there have been few attempts to compare the performance and cost of different first stage reuse strategies under common assumptions.

This paper attempts to fill that gap by evaluating, under a common framework, the relative merits of all major first stage reuse strategies. First, a taxonomy is developed which classifies first stage reuse strategies according to four architecture choices. Second, the effect of first stage reuse on launch vehicle payload capacity is considered and modeled. The payload performance model is derived from physical first principles and calibrated against historical launch vehicle data. Third, the development, production, and operation costs of each strategy are modeled using TRANSCOST 8.2. This is a high-level set of cost estimating relationships based on system masses, and calibrated against historical data. Finally, the payload performance and cost of several strategies are compared. We identify trade-offs between the strategies, and examine the variation of their performance and cost with propellant choice, the target orbit, payload size, launch rate, and first stage lifetime. We focus our analysis on orbital launch vehicles with two (sequential) stages, using established propulsion and structural technologies.

Our modeling approach emphasizes the representation of uncertainty and reproducibility. As a preliminary concept, point estimates of system performance and cost do not have much credibility - too much is unknown. Instead, we estimate distributions of payload and cost, which illustrate the credible range of outcomes. To do so, we quantify the uncertainty on our model parameters (e.g. by examining the spread of outcomes of past projects), and then use Monte Carlo techniques to determine the resulting uncertainty in the model outputs. To allow other workers to reproduce these efforts, we 1) use only publicly available data in our models and 2) release the software used to evaluate our models under an open license (TODO github link).


\bibliography{first_stage_recovery}

\end{document}
